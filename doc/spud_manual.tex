\documentclass[a4paper, 11pt]{book}

\usepackage{palatino}

\title{Spud 1.1.3 Manual}

\usepackage{listings}
\usepackage{hyperref,xspace}
\usepackage[margin=2cm]{geometry}
\usepackage{longtable}
\usepackage{color}

\definecolor{DarkBlue}{rgb}{0.00,0.00,0.55}
\hypersetup{
    linkcolor   = DarkBlue,
    anchorcolor = DarkBlue,
    citecolor   = DarkBlue,
    filecolor   = DarkBlue,
    pagecolor   = DarkBlue,
    urlcolor    = DarkBlue,
    colorlinks  = true,
    pdftitle    = {Spud 1.1.3 Manual},
}

\lstloadlanguages{Fortran,C++,C}

\lstset{basicstyle=\ttfamily}

\lstdefinelanguage{rnc}
{morekeywords={element,attribute},
  sensitive=false,
  morecomment=[l]{\#},
  morestring=[b]",
}

\newcommand{\fortran}[1]{\framebox{#1}}
\renewcommand{\c}[1]{\framebox{#1}}
\newcommand{\cpp}[1]{\framebox{#1}}

\newcommand{\stat}{\lstinline[language=fortran]+stat+\xspace}

\setlength{\parindent}{0pt}
\setlength{\parskip}{1ex}

\begin{document}
\maketitle

\tableofcontents


\chapter{Installation and building}

\section{Dependencies}

Spud and Diamond depend on the following packages:

\begin{itemize}
\item Fortran, C and C++ compilers
\item Python (\url{http://python.org/})
\item Python setuptools (\url{http://peak.telecommunity.com/DevCenter/setuptools})
\item PyGTK (\url{http://www.pygtk.org/})
\item lxml (\url{http://codespeak.net/lxml/})
\item Trang (\url{http://www.thaiopensource.com/relaxng/trang.html})
\item libxml2 (\url{http://xmlsoft.org/})
\item LaTeX (for the manual) (\url{http://www.latex-project.org/})
\end{itemize}

Users of Ubuntu should
be able to install all the needed dependencies by adding the repository listed below and typing:

\begin{verbatim}
sudo apt-get update
sudo apt-get build-depend spud
\end{verbatim}

\section{Obtaining the source}

Spud source code is held in a bzr repository. The current development version is available via:

\begin{verbatim}
bzr co lp:spud
\end{verbatim}

\section{Building from source}

Spud is built using a standard autoconf system. It should be possible to
build spud and install it in \verb+/usr/local+ simply by typing:

\begin{verbatim}
./configure
make
make install
\end{verbatim}

Installing to an alternative location is possible by specifying the
\verb+--prefix+ option to \verb+configure+. For a full list of configure
options type:

\begin{verbatim}
./configure --help
\end{verbatim}

\section{Ubuntu packages}

Ubuntu packages are available from a Launchpad repository managed by the Applied Modelling and Computation
Group at Imperial College London. To add this repository to your system sources, run:

\begin{verbatim}
sudo apt-add-repository ppa:amcg/ppa
\end{verbatim}

The Spud library and the base language are installed by the
\verb+libspud-dev+ package while diamond is shipped in the \verb+diamond+
package. Both packages are built from the \verb+spud+ source package. Binary
packages are supplied for the i386 and amd64 architectures.

After adding the repository, install the
packages by typing:

\begin{verbatim}
sudo apt-get update
sudo apt-get install diamond libspud-dev
\end{verbatim}

\chapter{Schemas and the Spud base language}
\lstset{language=rnc}

The World Wide Web Consortium's Extensible Markup Language (XML) provides a
generic syntax for machine parseable languages.  These allow the
organisation of model input options into a tree of nested elements.
Utilising such a structure within the options file allows distinct groups of
options to be gathered together in branches while suboption dependencies can
be represented as child elements.

\section{RELAX NG}

Spud uses the RELAX NG schema language within the XML system. For full
documentation of RELAX NG see \url{http://relaxng.org/}. The
\href{http://relaxng.org/compact-tutorial-20030326.html}{compact syntax
  tutorial}\ is particularly useful.

The examples presented here are shown in a compact syntax of RELAX NG.  This
is the preferred syntax for editing Spud schemas and the format it is
shipped in.  However the more verbose XML syntax is better supported by
software parsers.  Hence the completed schema, including the base language,
is translated from compact to XML syntax using the software package Trang
before use by Spud based tools like Diamond.

\section{Base language named patterns}\label{sec:named_patterns}

The RELAX NG language allows different schemas to be imported into one
another, which enables Spud to define a base language for schema developers.
The Spud base language thus provides core schema objects (known in RELAX NG
as patterns) that enable generic tools included in Spud to handle low level
data in an elegant manner.

For example the \lstinline*real_dim_symmetric_tensor* pattern is defined below:
\begin{lstlisting}
# A dim x dim real matrix (rank 2 tensor) constrained to be symmetric.
real_dim_symmetric_tensor =
   (
      element real_value{
         attribute symmetric {"true"},
         attribute rank { "2" },
         # Setting dim1, dim2 to a function of dim allows the gui
         # to set the tensor to the right shape.
         attribute dim1 { "dim" },
         attribute dim2 { "dim" },
         attribute shape { list{xsd:integer, xsd:integer} },
         list {xsd:float+}
      },
      comment
   )
\end{lstlisting}
This core object contains all the information required to define the
properties of a real, symmetric, rank 2 tensor with square dimensions equal
to the physical dimension specified.  This enables Spud based generic tools
to reduce the level of information required as input from the user.  In this
case the user is only required to provide a list of reals to fill out the
tensor while the generic tool will ensure it is symmetric, ordered correctly
and save the rank, shape and dimensions.  Thus from the developers
perspective all the information required to import a symmetric tensor is
available from Spud alongside the user's input.

As can be seen above the principal element of the \lstinline
*real_dim_symmetric_tensor* pattern is \lstinline*real_value*.
Modification of the rank, shape and dimension attributes of this element
allows the Spud based language to be expanded easily to incorporate other
real data structures.  For instance, a rank 1 real vector with length equal
to the physical dimension specified is defined in the \lstinline
*real_dim_vector* pattern:
\begin{lstlisting}
# A real vector of length dim
real_dim_vector =
   (
      element real_value{
         attribute rank { "1" },
         # Setting dim1 to a function of dim allows the gui to set the
         # vector to the right length.
         attribute dim1 { "dim" },
         attribute shape { xsd:integer },
         list{xsd:float+}
      },
      comment
   )
\end{lstlisting}

Similar extensions can be made for an integer based \lstinline*integer_value* element, while a \lstinline*string_value* element allows the definition of several character patterns known to Spud generic tools (such as the \lstinline*comment* pattern seen in the examples above).  This allows for the definition of the full Spud base language as follows:

\noindent\begin{longtable}{lp{8cm}}
\lstinline*comment* & A string value that allows users to annotate their input throughout the XML tree structure.  Included in all Spud base language patterns. \\
\lstinline*anystring* & A string value for the input of any generic character string required in the options tree.  Suggests a display of 1 line within Spud based tools such as Diamond. \\
\lstinline*filename* & A string value for the input of filenames in the options tree.  Allows the use of a file selector and suggests a display of 1 line within Spud based tools such as Diamond. \\
\lstinline*Python_code* & A string value for the input of Python code into the options tree.  Suggests a display of 20 lines within Spud based tools such as Diamond. \\
\lstinline*integer* & An integer value of rank 0 and length 1. \\
\lstinline*integer_vector* & A rank 1 vector of integers of arbitrary length.  Spud tools record the shape of the input. \\
\lstinline*integer_tensor* & A tensor of integers of arbitrary dimensions.  Spud tools record the shape of the input. \\
\lstinline*integer_dim_vector* & A rank 1 vector of integers with length equal to the physical dimension specified. \\
\lstinline*integer_dim_minus_one_vector* & A rank 1 vector of integers with length equal to 1 less than the physical dimension specified. \\
\lstinline*integer_dim_tensor* & A rank 2 tensor of integers with square dimensions equal to the physical dimension specified. \\
\lstinline*integer_dim_symmetric_tensor* & A rank 2 symmetric tensor of integers with square dimensions equal to the physical dimension specified. \\
\lstinline*integer_dim_minus_one_tensor* & A rank 2 tensor of integers with square dimensions equal to 1 less than the physical dimension specified. \\
\lstinline*integer_dim_minus_one_symmetric_tensor* & A rank 2 symmetric tensor of integers with square dimensions equal to 1 less than the physical dimension specified. \\
\lstinline*real* & A real value of rank 0 and length 1. \\
\lstinline*real_vector* & A rank 1 vector of reals of arbitrary length.  Spud tools record the shape of the input. \\
\lstinline*real_tensor* & A rank 2 tensor of reals of arbitrary dimensions.  Spud tools record the shape of the input. \\
\lstinline*real_dim_vector* & A rank 1 vector of reals with length equal to the physical dimension specified. \\
\lstinline*real_dim_minus_one_vector* & A rank 1 vector of reals with length equal to 1 less than the physical dimension specified. \\
\lstinline*real_dim_tensor* & A rank 2 tensor of reals with square dimensions equal to the physical dimension specified. \\
\lstinline*real_dim_symmetric_tensor* & A rank 2 tensor of reals with square dimensions equal to the physical dimension specified. \\
\lstinline*real_dim_minus_one_tensor* & A rank 2 tensor of reals with square dimensions equal to 1 less than the physical dimension specified. \\
\lstinline*real_dim_minus_one_symmetric_tensor* & A rank 2 tensor of reals with square dimensions equal to 1 less than the physical dimension specified.
\end{longtable}

\subsection{Including the base language in the schema}
The base language is described in the file \verb+spud_base.rnc+.
This file is installed in \verb+@prefix@/share/spud+, where the default
value of \verb+@prefix@+ is \verb+/usr+. It may be referenced
without prefixing the path. The top line of every spud schema should read
\begin{verbatim}
include "spud_base.rnc"
\end{verbatim}

\subsection{The real\_value and integer\_value elements}

As mentioned above, the \verb+real_value+ and \verb+integer_value+
elements are the low-level implementations of the real and integer named
patterns in the base language. The full schema syntax of these elements is
illustrated by the definition of the \verb+real_dim_minus_one_tensor+
pattern in the base language:
\begin{lstlisting}
# A dim-1 x dim-1 real matrix (rank 2 tensor).
real_dim_minus_one_tensor =
   (
      element real_value{
         attribute symmetric {"false"},
         attribute rank { "2" },
         # Setting dim1, dim2 to a function of dim allows the gui to set the
         # tensor to the right shape.
         attribute dim1 { "dim-1" },
         attribute dim2 { "dim-1" },
         attribute shape { list{xsd:integer, xsd:integer} },
         list {xsd:float+}
      },
      comment
   )
\end{lstlisting}
The \lstinline+symmetric+ and \lstinline+dim2+ attributes are only present
if \lstinline+rank+ is 2 while the \lstinline+dim1+ and \lstinline+shape+
attributes are only present where \lstinline+rank+ is at least 1. If
\lstinline+rank+ is equal to 1 then \lstinline+shape+ will be a single
\lstinline+xsd:integer+ rather than a list of 2. The \lstinline+dim1+ and
\lstinline+dim2+ attributes are Python expressions of the variable
\lstinline+dim+.


\subsection{The string\_value element}

Customised string values, such as fixed strings or multiline strings, are
constructed using the \lstinline+string_value+ element. This element wraps a
string and a \lstinline+lines+ attribute. The \lstinline+lines+ attribute
does not enforce a length on the string but is rather a hint to the user
interface as to the size of string box which would be appropriate. The
schema syntax of the \lstinline+string_value+ element is illustrated by the
definition of the main \lstinline+anystring pattern+ from the base language:
\begin{lstlisting}
# A simple string
anystring =
   (
      element string_value{
         # Lines is a hint to the gui about the size of the text box.
         # It is not an enforced limit on string length.
         attribute lines { "1" },
         xsd:string
      },
      comment
   )
\end{lstlisting}
An example of a customised string is this choice of two available string
values:
\begin{lstlisting}
## Format for dump files. Choose from fluidity dumpfile or vtk.
element dump_format {
   element string_value{
      "fluidity dumpfile"|"vtk"
   }
}  
\end{lstlisting}

\subsection{Symmetric tensors}

Some of the Spud base language patterns refer to symmetric tensors. In every
case, the full tensor is stored and retrieving the tensor using libspud. It
is the responsibility of Spud user interface tools such as Diamond to
enforce the symmetry of tensors input by the user.

\section{Specifying problem dimension }

The dimension of the problem affects how all of the dimension-specific named
patterns in the schema are handled. It is specified by having an
integer-valued \lstinline+dimension+ element as a child of the
\lstinline+geometry+ element which is in turn a child of the root
element. Spud schemas are required to define \lstinline+dimension+ and
\lstinline+geometry+ elements if they make use of any of the
named patterns whose name includes the string \verb+dim+. Figure
\ref{fig:schema}\ illustrates a trivial Spud schema in which the dimension
attribute is given using a customised \lstinline+integer_value+ element
which only permits the values 2 or 3 to be given.

\begin{figure}[t]
\begin{lstlisting}[language=rnc,frame=trBL]
include "spud_base.rnc"

start = 
   (
      # Outside wrapper element. Doesn't really matter what the name is.
      element model_options {
          comment,
         ## Model output files are named according to the simulation name, 
         ## e.g. [simulation_name]_0.vtu. Non-standard characters in the 
         ## simulation name should be avoided.
         element simulation_name {
            anystring
         },
         ## Options dealing with the specification of geometry
         element geometry {
            ## Dimension of the problem.
            ## <b>This can only be set once</b>
            element dimension {
               attribute replaces {"NDIM"},
               element integer_value {
                  attribute rank {"0"},
                  ("3"|"2")
               }
            }
         }
      }
   )  
\end{lstlisting}

  \caption{A trivial schema showing a schema comment, schema annotations and
  a user comment pattern. The dimension of the problem is specified by the
  dimension element under the geometry element.}
  \label{fig:schema}

\end{figure}

\section{Restrictions on the base language}

In order to facilitate processing of schemas by libspud and Diamond, and in
particular to make the presentation of a simple and intuitive user interface
possible, several restrictions are imposed on RELAX NG schemas used in Spud.

\begin{itemize}
\item Choice nodes must be choices between single elements (e.g. a choice
  between a OR (b AND c) is invalid).
\item Elements with the same tag under the same parent are allowed. However,
  at most one can be + (oneOrMore) or * (zeroOrMore). Elements with the same
  tag under the same parent must each have a name attribute with a unique
  value.
\item \lstinline+name+ attributes may contain only alpha-numeric characters,
  or characters in the set "\verb+/_:[]+"
\item Recursive schema elements are not supported.
\end{itemize}

\section{Comments and annotations}

There are three layers of comment which are applicable in Spud. The schema,
as with any piece of source code, can contain comments which are of use to
other developers editing the schema. Second, the schema can embed
documentation for the problem description language. This documentation will
be displayed to the model user by Diamond. The former comments are known as
schema comments and are written with a single leading hash (\verb+#+)
while the latter are known as schema annotations and are written with a
double leading hash (\verb+##+). Schema annotations must be written
immediately before the element they document. 

Finally, the \verb+comment+ named pattern will cause diamond to associate a
user comment box with the parent element. This enables users to document
their problem description files. Each of the named patterns in the base
language also includes the \verb+comment+ pattern so that every parameter in
an input file can have a user comment associated with it. Figure
\ref{fig:schema}\ shows a simple schema incorporating all three layers of
comment.


\section{Preprocessing the schema for use with Diamond}
RELAX NG comes in two equivalent formats: XML syntax (with
file suffix \verb+.rng+) and compact
syntax (with file suffix \verb+.rnc+).
Compact syntax is optimised for human use, while XML
syntax is optimised for ease of machine parsing. Therefore,
it is recommended that model developers write the schema
in compact syntax, then transform it using a supplied tool
to XML syntax for use with Diamond and other validation tools.
To transform compact syntax into XML syntax, use the command:
\begin{verbatim}
spud-preprocess /path/to/schema.rnc
\end{verbatim}
This will create a file called schema.rng in the same directory.

\chapter{Libspud}

Libspud provides C, C++ and Fortran interfaces for accessing the options
specified in a Spud XML file.

\section{The options tree}

Spud XML files are read into an in-memory tree structure which reflects the
tree of nested elements in the XML. Nodes in this tree are indexed by
strings, i.e. the options tree is a dictionary in which values are
interrogated via keys.

\section{Option key syntax}

The option key syntax is similar to Unix file path syntax.
Consider the simple example, \verb+simple.xml+, which is valid with respect
to the schema in figure \ref{fig:schema}:
\begin{verbatim}
<model_options>
  <simulation_name>
    <string_value lines="1">Basic simulation</string_value>
  </simulation_name>
  <geometry>
    <dimension>
      <integer_value rank="0">3</integer_value>
    </dimension>
  </geometry>
</model_options>
\end{verbatim}

The simulation name and the dimension of the geometry may be accessed
using the following Fortran program:
\begin{verbatim}
program fetch_info
  use spud
  implicit none

  integer :: dimension
  character(len=255) :: simulation_name

  call load_options("simple.xml")
  call get_option("/simulation_name", simulation_name)
  call get_option("/geometry/dimension", dimension)
end program fetch_info
\end{verbatim}

The option key \verb+/simulation_name+ accesses the element
called \verb+simulation_name+ that is a child of the root element
(which in this case is called \verb+model_options+). The option key
\verb+/geometry/dimension+ accesses the element \verb+dimension+ which
in turn is a child of the root element.

\subsection{Multiple elements}
One of the restrictions that Spud places on RELAX NG schemas
is that \emph{each element of the same name under the same parent
must be differentiated by a \texttt{name} attribute}. An example will
clarify what is valid. Consider the file \verb+complex_invalid.xml+:
\begin{verbatim}
<model_options>
  <simulation_name>
    <string_value lines="1">Basic simulation</string_value>
  </simulation_name>
  <geometry>
    <dimension>
      <integer_value rank="0">3</integer_value>
    </dimension>
    <mesh>
      <string_value type="filename" lines="1">mesh_A.msh</string_value>
    </mesh>
    <mesh>
      <string_value type="filename" lines="1">mesh_B.msh</string_value>
    </mesh>
  </geometry>
</model_options>
\end{verbatim}

This file is invalid as there are two \verb+mesh+ elements beneath the same
\verb+geometry+ element, and they are not differentiated by a unique \verb+name+
attribute. To make this file valid we must differentiate the two \verb+mesh+
elements by adding a \verb+name+ attribute:

\begin{verbatim}
<model_options>
  <simulation_name>
    <string_value lines="1">Basic simulation</string_value>
  </simulation_name>
  <geometry>
    <dimension>
      <integer_value rank="0">3</integer_value>
    </dimension>
    <mesh name="PositionMesh">
      <string_value type="filename" lines="1">mesh_A.msh</string_value>
    </mesh>
    <mesh name="VelocityMesh">
      <string_value type="filename" lines="1">mesh_B.msh</string_value>
    </mesh>
  </geometry>
</model_options>
\end{verbatim}

This file is now valid as the two \verb+mesh+ elements have different
\verb+name+ values.

To access these elements in the model, the option keys
\begin{verbatim}
/geometry/mesh[0]
/geometry/mesh[1]
\end{verbatim}
may be used to access the elements in order, or they may be accessed
by \verb+name+ by
\begin{verbatim}
/geometry/mesh::PositionMesh
/geometry/mesh::VelocityMesh
\end{verbatim}

\subsection{Attributes}

For simplicity, attributes of an element are treated the same as children
of the element. Consider the example above: \verb+name+ is an attribute
of the \verb+mesh+ element. The \verb+name+ of the first \verb+mesh+ element
may be accessed by
\begin{verbatim}
/geometry/mesh[0]/name
\end{verbatim}
that is, the \verb+name+ attribute is accessed the same way as a child element
called \verb+name+ would be.

Attributes in the options tree must have string type data and have no children.
If either of these rules are broken (e.g. via a \ref{sec:set_option} call)
then the attribute element will be unmarked as an attribute, will be treated as
a normal element, and will appear in XML files written out by libspud as XML
elements rather than element attributes.

\subsection{Data elements}

Data defined in the Spud base language (see \ref{sec:named_patterns}), such as
\verb+integer_value+ or \verb+real_value+, are detected by libspud and stored in
"\verb+__value+" children. e.g, for the following schema:

\begin{verbatim}
<model_options>
  <real_parent>
    <ref name="real_value"/>
  </real_parent>
</model_options>
\end{verbatim}

the data element "\verb+real_value+" has option key
"\verb+/real_parent/__value+". Alternatively, the data element
"\verb+real_value+" can be accessed directly with option key
"\verb+/real_parent+" - i.e. libspud automatically navigates into
"\verb+__value+" child elements when reading or setting options, if such a child
element exists.

Manually creating a "\verb+__value+" child for an element that already itself
contains data will result in the data of the parent element being removed, and
a warning message will be sent to standard error.


\section{Language specific features}

In some cases the interfaces differ slightly between Fortran, C and C++.
This is brought about by differences in the designs of these languages
themselves and in particular the difference in the manner in which optional
arguments are supported in Fortran on the one hand and C/C++ on the other.
The decision has been made to write the interfaces in the manner which seems
natural in each language at the cost of consistency between languages rather
than enforcing a foreign paradigm on one or all of the interfaces.

\subsection{Fortran}

All of the Fortran procedures as well as the named constants for error codes
(\ref{sec:error_codes}) and data types
(\ref{sec:types})  are encapsulated in the \lstinline+spud+ module.

Where a routine returns an error code, in Fortran this is achieved via the
optional \stat argument. If \stat is not present and an error code other
than \lstinline[language=fortran]+SPUD_NO_ERROR+ is returned then execution
will halt with an error message.

\subsection{C}

Since C does not itself have any namespacing facility, all exposed symbols
have the prefix \lstinline[language=C]+spud_+. Error codes are returned via
function return values.

\subsection{C++}

The entire public C++ API of libspud is contained in the
\lstinline[language=C++]+Spud+ namespace. Error codes are returned via function
return values.

\section{Naming conventions}

Where a routine returns its main result via an argument (as is the case for
a Fortran subroutine, for example), the routine's name starts with
\lstinline[language=fortran]+get_+. The word
\lstinline[language=fortran]+key+ is exclusively used to refer to a lookup key in
the options dictionary.

\section{Procedure interfaces}
\lstset{frame=single}

In each case, the Fortran interface is given first, followed by the C and
then C++ interfaces.

\subsection{Error codes}\label{sec:error_codes}

The following values are return statuses of procedures. In Fortran these are
named constants in the \lstinline+spud+ module while in C and C++ these are
the enum types \lstinline+SpudOptionError+ and \lstinline+Spud::OptionError+
respectively.

Error values are greater than zero, warnings are negative and
\lstinline+SPUD_NO_ERROR+ has the value 0.

\begin{tabular}{lp{8cm}}
  \textbf{Error code} & \textbf{Interpretation}\\
  \lstinline+SPUD_NO_ERROR+ & Successful completion.\\
  \lstinline+SPUD_KEY_ERROR+ & The specified option is not present in the
  dictionary.\\
  \lstinline+SPUD_TYPE_ERROR+ & The specified option has a different type
  from that of the option argument provided.\\
  \lstinline+SPUD_RANK_ERROR+ & The specified option has a different rank
  from that of the option argument provided.\\
  \lstinline+SPUD_SHAPE_ERROR+ & The specified option has a different shape
  from that of the option argument provided.\\
  \lstinline+SPUD_FILE_ERROR+ & The specified options file cannot be read or
  written to as the routine requires.\\
  \lstinline+SPUD_NEW_KEY_WARNING+ & The option being inserted is not
  already in the dictionary.\\
  \lstinline+SPUD_ATTR_SET_FAILED_WARNING+ & The option being set as an
  attribute can not be set as an attribute.
\end{tabular}

\subsection{Data type parameters}\label{sec:types}

The option\_type routine returns the following values. In Fortran these are
named constants in the \lstinline+spud+ module while in C and C++ these are
the enum types \lstinline+SpudOptionType+ and \lstinline+Spud::OptionType+
respectively.

\begin{tabular}{ll}
  \textbf{Fortran} & \textbf{C/C++} \\
  \lstinline+SPUD_REAL+ & \lstinline+SPUD_DOUBLE+\\
  \lstinline+SPUD_INTEGER+ & \lstinline+SPUD_INT+\\
  \lstinline+SPUD_NONE+ & \lstinline+SPUD_NONE+\\
  \lstinline+SPUD_CHARACTER+ & \lstinline+SPUD_STRING+
\end{tabular}

\subsection{clear\_options}

\begin{lstlisting}[language=fortran]
subroutine clear_options()
end subroutine clear_options
\end{lstlisting}

\begin{lstlisting}[language=C]
void spud_clear_options()
\end{lstlisting}

\begin{lstlisting}
void Spud::clear_options();
\end{lstlisting}

Clears the entire options tree.

\subsection{load\_options}

\begin{lstlisting}[language=fortran]
function load_options(filename, stat)
  character(len=*), intent(in) :: filename
  integer, optional, intent(out) :: stat
\end{lstlisting}

\begin{lstlisting}[language=C]
int spud_load_options(const char* key, const int key_len)
\end{lstlisting}

\begin{lstlisting}
OptionError Spud::load_options(const std::string& filename)
\end{lstlisting}

Reads the XML file \lstinline+filename+ into the options tree.

Returns error code \lstinline+SPUD_FILE_ERROR+ if the file does not exist or cannot be read.

\subsection{write\_options}

\begin{lstlisting}[language=fortran]
subroutine write_options(filename, stat)
  character(len=*), intent(in) :: filename
  integer, optional, intent(out) :: stat
\end{lstlisting}

\begin{lstlisting}[language=C]
int spud_write_options(const char* filename, const int filename_len)
\end{lstlisting}

\begin{lstlisting}[language=C++]
OptionError write_options(const std::string& filename)
\end{lstlisting}

Writes the options tree out to the XML file \lstinline+filename+.

Returns error code \lstinline+SPUD_FILE_ERROR+ if the file does not exist or cannot be written.

\subsection{get\_child\_name}

\begin{lstlisting}[language=fortran]
subroutine get_child_name(key, index, child_name)
  character(len=*), intent(in)::key
  integer, intent(in)::index
  character(len=*), intent(out)::child_name
\end{lstlisting}

\begin{lstlisting}[language=C]
int spud_get_child_name(const char* key, const int key_len,
  const int index,
  char* child_name, const int child_name_len)
\end{lstlisting}

\begin{lstlisting}[language=C++]
Spud::OptionError Spud::get_child_name(const std::string& key,
  const unsigned& index,
  std::string& child_name)
\end{lstlisting}

Retrieves the name of the \lstinline+index+th child of \lstinline+key+. This
is mostly useful for debugging input files.

Returns error code \lstinline+SPUD_KEY_ERROR+ if the supplied key does not
exist in the options tree.

\subsection{get\_number\_of\_children}

\begin{lstlisting}[language=Fortran]
function number_of_children(key, child_count)
  integer :: number_of_children
  character(len=*), intent(in) :: key
  integer, intent(out) :: child_count
  integer, optional, intent(out) :: stat
\end{lstlisting}

\begin{lstlisting}[language=C]
int spud_get_number_of_children(const char* key, const int key_len, int* child_count)
\end{lstlisting}

\begin{lstlisting}[language=C++]
int Spud::get_number_of_children(const std::string& key, int& child_count)
\end{lstlisting}

On return, \lstinline+child_count+  the number of children under \lstinline+key+. This is mainly of use
for debugging input files.

Returns the error code \lstinline+SPUD_KEY_ERROR+ if the specified key does not exist in the options tree.

\subsection{option\_count}

\begin{lstlisting}[language=fortran]
function option_count(key)
  integer :: option_count
  character(len=*), intent(in) :: key
\end{lstlisting}

\begin{lstlisting}[language=C]
int spud_option_count(const char* key, const int key_len)
\end{lstlisting}

\begin{lstlisting}[language=C++]
int Spud::option_count(const std::string& key)
\end{lstlisting}

Returns the number of options which match \lstinline+key+. Searches all possible
paths matching the given key. For example, for the following XML file:

\begin{verbatim}
<model_options>
  <scalar_field name="pressure">
    <solver>
      ...
    </solver>
  </scalar_field>
  <scalar_field name="temperature">
    <solver>
      ...
    </solver>
  </scalar_field>
</model_options>
\end{verbatim}

in the following Fortran code:

\begin{lstlisting}[language=fortran]
n_child = option_count("/scalar_field/solver")
\end{lstlisting}

\lstinline+n_child+ is assigned the value 2.

This routine is useful where an option can occur any number of times (for
example a simulation may allow for any number of fields to be specified).

Returns 0 if \lstinline+key+ is not present in the dictionary.

\subsection{have\_option}

\begin{lstlisting}[language=fortran]
function have_option(key)
  logical :: have_option
  character(len=*), intent(in) :: key
\end{lstlisting}

\begin{lstlisting}[language=C]
int spud_have_option(const char* key, const int key_len)
\end{lstlisting}

\begin{lstlisting}[language=C++]
logical_t Spud::have_option(const std::string& key)
\end{lstlisting}

Returns true if \lstinline+key+ is present in the options dictionary, and
false otherwise. This is useful for determining whether optional options have
been set and for determining which of a choice of options has been selected.

\subsection{option\_type}\label{sec:option_type}

\begin{lstlisting}[language=fortran]
function option_type(key, stat) result (type)
  integer :: type
  character(len=*), intent(in) :: key
  integer, optional, intent(out) :: stat
\end{lstlisting}

\begin{lstlisting}[language=C]
int spud_get_option_type(const char* key, const int key_len, int* type)
\end{lstlisting}

\begin{lstlisting}[language=C++]
Spud::OptionError Spud::get_option_type(const std::string& key,
Spud::OptionType& type)
\end{lstlisting}

Returns the type of the option specified by \lstinline+key+. The type will
be returned as one of the named constants in \ref{sec:types}.

Returns error code \lstinline+SPUD_KEY_ERROR+ if the supplied key does not
exist in the options tree.

\subsection{option\_rank}

\begin{lstlisting}[language=fortran]
function option_rank(key, stat) result (rank)
  integer :: rank
  character(len=*), intent(in) :: key
  integer, optional, intent(out) :: stat
\end{lstlisting}

\begin{lstlisting}[language=C]
int spud_get_option_rank(const char* key, const int key_len, int* rank)
\end{lstlisting}

\begin{lstlisting}[language=C++]
Spud::OptionError Spud::get_option_rank(const std::string& key,
int& rank)
\end{lstlisting}

Returns the rank of the option specified by \lstinline+key+. The rank returned
will be 0 (for a scalar), 1 (for a vector) or 2 (a rank 2 tensor, or matrix).

Returns error code \lstinline+SPUD_KEY_ERROR+ if the supplied key does not
exist in the options tree.

\subsection{option\_shape}

\begin{lstlisting}[language=fortran]
function option_shape(key, stat) result (lshape)
  integer, dimension(2) :: lshape
  character(len=*), intent(in) :: key
  integer, optional, intent(out) :: stat
\end{lstlisting}

\begin{lstlisting}[language=C]
int spud_get_option_shape(const char* key, const int key_len, int* shape)
\end{lstlisting}

\begin{lstlisting}[language=C++]
Spud::OptionError Spud::get_option_shape(const std::string& key,
std::vector<int>& shape)
\end{lstlisting}

Returns the shape of the option specified by \lstinline+key+. The shape is
always a 2-vector. If the option in question is rank 1 then the second
component will be -1, if the option is rank 0 (a scalar) then both entries
will be -1.

Returns error code \lstinline+SPUD_KEY_ERROR+ if the supplied key does not
exist in the options tree.

\subsection{get\_option}\label{sec:get_option}

\begin{lstlisting}[language=fortran,emph=option_type,emphstyle=\textit]
subroutine get_option(key, val, stat, default)
  character(len=*), intent(in) :: key
  option_type, intent(out) :: val
  integer, optional, intent(out) :: stat
  option_type, optional, intent(in) :: default
\end{lstlisting}

\begin{lstlisting}[language=C,emph=option_type,emphstyle=\textit]
int spud_get_option(const char* key, const int key_len, void* val)
\end{lstlisting}

\begin{lstlisting}[language=C++,emph=option_type,emphstyle=\textit]
Spud::OptionError Spud::get_option(const std::string& key,
option_type val)

Spud::OptionError Spud::get_option(const std::string& key,
option_type val, option_type default_val)
\end{lstlisting}

This is the main method for retrieving option values from the options
dictionary. For Fortran and C++
\lstinline[emph=option_type,emphstyle=\textit]+option_type+ can be any of the
following values:

\begin{tabular}{ll}
  \textbf{Fortran type} & \textbf{C++ type} \\
   \lstinline[language=fortran]+double precision+ &
   \lstinline[language=C++]+double&+ \\
   \lstinline[language=fortran]+double precision, dimension(:)+ &
   \lstinline[language=C++]+std::vector<double>&+ \\
   \lstinline[language=fortran]+double precision, dimension(:,:)+ &
   \lstinline[language=C++]+std::vector< std::vector<double> >&+\\
   \lstinline[language=fortran]+integer+ &
   \lstinline[language=C++]+int&+ \\
   \lstinline[language=fortran]+integer, dimension(:)+ &
   \lstinline[language=C++]+std::vector<int>&+ \\
   \lstinline[language=fortran]+integer, dimension(:,:)+ &
   \lstinline[language=C++]+std::vector< std::vector<int> >&+\\
   \lstinline[language=fortran]+character(len=*)+ &
   \lstinline[language=C++]+std::string&+
\end{tabular}

In Fortran, single precision interfaces are also provided in each of the
above cases. However, these values will be stored in the options dictionary
in double precision. In every case the type and shape of the argument must
match that of the option in the dictionary. In C the returned argument is
always a \lstinline+void+ pointer and it is the responsibility of the user
to match the size and type correctly.

This routine can return the following error codes:
\begin{itemize}
\item If \lstinline+key+ matches an option but the shape, rank or type fails
  to match, appropriate error  code will be set (see
  \ref{sec:error_codes}).
\item If \lstinline+key+ fails to match but \lstinline+default+ is present,
  \lstinline+option+ is set to the value of \lstinline+default+.
\item If \lstinline+key+ fails to match and \lstinline+default+ is not
  present, the error code will be set to \lstinline+SPUD_KEY_ERROR+.
\end{itemize}

\subsection{add\_option}

\begin{lstlisting}[language=fortran]
subroutine add_option(key, stat)
  character(len=*), intent(in) :: key
  integer, optional, intent(out) :: stat
\end{lstlisting}

\begin{lstlisting}[language=C]
int spud_add_option(const char* key, const int key_len)
\end{lstlisting}

\begin{lstlisting}[language=C++]
Spud::OptionError Spud::add_option(const std::string& key)
\end{lstlisting}

Creates a new option at the supplied key. If the option does not currently
exist, creates the option (with data type \lstinline+SPUD_NONE+) and
returns error code \lstinline+SPUD_NEW_KEY_WARNING+.

\subsection{set\_option}\label{sec:set_option}

\begin{lstlisting}[language=fortran,emph=option_type,emphstyle=\textit]
subroutine set_option(key, val, stat)
  character(len=*), intent(in) :: key
  option_type, intent(in or inout) :: val
  integer, optional, intent(out) :: stat
\end{lstlisting}

\begin{lstlisting}[language=C,emph=option_type,emphstyle=\textit]
int spud_set_option(const char* key, const int key_len, const void* val,
const int type, const int rank, const int* shape)
\end{lstlisting}

\begin{lstlisting}[language=C++,emph=option_type,emphstyle=\textit]
Spud::OptionError Spud::set_option(const std::string& key,
const option_type& val)
\end{lstlisting}

Method for setting options in the options tree. The
\lstinline[emph=option_type,emphstyle=\textit]+option_type+ can be any of the
types listed above in \ref{sec:get_option}.

This routine can return the following error codes:
\begin{itemize}
\item If \lstinline+key+ matches an option but the shape, rank or type fails
  to match currently existing option, returns an appropriate error code (see
  \ref{sec:error_codes}).
\item If \lstinline+key+ fails to match, creates a new option at the supplied
  key, sets the option to \lstinline+val+ and returns error code
  \lstinline+SPUD_NEW_KEY_WARNING+.
\end{itemize}

\subsection{set\_option\_attribute}

\begin{lstlisting}[language=fortran]
subroutine set_option_attribute(key, val, stat)
  character(len=*), intent(in) :: key
  character(len=*), intent(in) :: val
  integer, optional, intent(out) :: stat
\end{lstlisting}

\begin{lstlisting}[language=C]
int int spud_set_option_attribute(const char* key, const int key_len,
const char* val, const int val_len)
\end{lstlisting}

\begin{lstlisting}[language=C++]
Spud::OptionError Spud::set_option_attribute(const std::string& key,
const std::string& val)
\end{lstlisting}

As \lstinline+set_option+ (see \ref{sec:set_option}), but additionally attempts
to mark the option at the specified key as an attribute. Note that
\lstinline+set_option_attribute+ accepts only string data data for
\lstinline+val+.

This routine can return the following error codes:
\begin{itemize}
\item If \lstinline+key+ matches an option but the shape, rank or type fails
  to match currently existing option, returns an appropriate error code (see
  \ref{sec:error_codes}).
\item If \lstinline+key+ fails to match, creates a new option at the supplied
  key, sets the option to \lstinline+val+ and returns error code
  \lstinline+SPUD_NEW_KEY_WARNING+.
\item If \lstinline+key+ matches an option, but the existing option has
  children, sets the option to \lstinline+val+ and returns error code
  \lstinline+SPUD_ATTR_SET_FAILED_WARNING+.
\end{itemize}

\subsection{delete\_option}

\begin{lstlisting}[language=fortran]
subroutine delete_option(key, stat)
  character(len=*), intent(in) :: key
  integer, optional, intent(out) :: stat
\end{lstlisting}

\begin{lstlisting}[language=C]
int spud_delete_option(const char* key, const int key_len)
\end{lstlisting}

\begin{lstlisting}[language=C++]
Spud::OptionError Spud::delete_option(const std::string& key)
\end{lstlisting}

Deletes the option at the specified key.

Returns error code \lstinline+SPUD_KEY_ERROR+ if the supplied key does not
exist in the options tree.

\subsection{move\_option}

\begin{lstlisting}[language=fortran]
subroutine move_option(key1, key2, stat)
  character(len=*), intent(in) :: key1
  character(len=*), intent(in) :: key2
  integer, optional, intent(out) :: stat
\end{lstlisting}

\begin{lstlisting}[language=C]
int spud_move_option(const char* key1, const int key1_len, const char* key2, const int key2_len)
\end{lstlisting}

\begin{lstlisting}[language=C++]
Spud::OptionError Spud::move_option(const std::string& key1, const std::string& key2)
\end{lstlisting}

Moves the entire options tree and all its children from key1 to key2.

\subsection{copy\_option}

\begin{lstlisting}[language=fortran]
subroutine copy_option(key1, key2, stat)
  character(len=*), intent(in) :: key1
  character(len=*), intent(in) :: key2
  integer, optional, intent(out) :: stat
\end{lstlisting}

\begin{lstlisting}[language=C]
int spud_copy_option(const char* key1, const int key1_len, const char* key2, const int key2_len)
\end{lstlisting}

\begin{lstlisting}[language=C++]
Spud::OptionError Spud::copy_option(const std::string& key1, const std::string& key2)
\end{lstlisting}

Copies the entire options tree and all its children from key1 to key2.

\subsection{print\_options}

\begin{lstlisting}[language=fortran]
subroutine print_options()
\end{lstlisting}

\begin{lstlisting}[language=C]
void spud_print_options()
\end{lstlisting}

\begin{lstlisting}[language=C++]
void Spud::print_options()
\end{lstlisting}

Prints the entire options tree to standard output. Useful for debugging.

\section{Python binding for libspud}

libspud also offers bindings for the Python programming language.  Users can use Python for accessing the options specified in a Spud XML file.  (This is done by the libspud.c module.  The module is written in C using the header file Python.h and spud.h.  It provides a Python interface to libspud; so that users could use Python codes to access the C codes in libspud.  The module takes in Python arguments, converts them into C arguments and then call the corresponding C functions with the converted C arguments.) 

\subsection{Errors}

Python binding converts spud errors into exceptions, and that these exceptions match those listed in 3.5.2:  \\* 
SpudError; \\* 
SpudTypeError; \\* 
SpudKeyError; \\* 
SpudFileError; \\* 
SpudNewKeyWarning; \\* 
SpudAttrSetFailedWarning; \\* 
SpudShapeError; \\* 
SpudRankError; 

\subsection{Data type parameters}\label{sec:types}

The option\_type routine returns the following values. In Fortran these are
named constants in the \lstinline+spud+ module while in C and C++ these are
the enum types \lstinline+SpudOptionType+ and \lstinline+Spud::OptionType+
respectively.

\begin{tabular}{lll}
  \textbf{Fortran} & \textbf{C/C++} & \textbf{Python} \\
  \lstinline+SPUD_REAL+ & \lstinline+SPUD_DOUBLE+ & \lstinline+float+ \\
  \lstinline+SPUD_INTEGER+ & \lstinline+SPUD_INT+ & \lstinline+int+ \\
  \lstinline+SPUD_NONE+ & \lstinline+SPUD_NONE+ & \lstinline+None+ \\
  \lstinline+SPUD_CHARACTER+ & \lstinline+SPUD_STRING+ & \lstinline+str+
\end{tabular}

\subsection{clear\_options}

\begin{lstlisting}[language=Python]
def clear_options()
return None
\end{lstlisting}

This function takes no arguments and returns None.
Clears the entire options tree.

\subsection{load\_options}

\begin{lstlisting}[language=Python]
def load_options(string filename)
return None
\end{lstlisting}

This function takes the XML filename in the form of a Python string and raises SpudFileError if file cannot be read or does not exist.
Otherwise, it reads the file into the options tree, and then returns None.

\subsection{write\_options}

\begin{lstlisting}[language=Python]
def write_options(string filename)
return None
\end{lstlisting}

This function takes the XML filename in the form of a Python string and raises SpudFileError if file cannot be written or does not exist.
Otherwise, it writes the options tree to the file, and then returns None.

\subsection{get\_child\_name}

\begin{lstlisting}[language=Python]
def get_child_name(string key, int index)
return string
\end{lstlisting}

This function takes the key in the form of a Python string, and an integer index. 
It raises SpudKeyError if supplied key does not exist in the options tree.
Otherwise, it retrieves the childname of the indexth of key , and then returns the childname as Python string.

\subsection{get\_number\_of\_children}

\begin{lstlisting}[language=Python]
def get_number_of_children(string key)
return int
\end{lstlisting}

This function takes the key in the form of a Python string. 
It raises SpudKeyError if supplied key does not exist in the options tree.
Otherwise, it returns the number of children of the key as Python integer.

\subsection{option\_count}

\begin{lstlisting}[language=Python]
def option_count(string key)
return int
\end{lstlisting}

This function takes the key in the form of a Python string. 
It returns the number of options which match the key as Python integer.
It returns 0 if key is not present in the dictionary.

\subsection{have\_option}

\begin{lstlisting}[language=Python]
def have_option(string key)
return True or False
\end{lstlisting}

This function takes the key in the form of a Python string. 
It returns True if the key is present in the options dictionary, False otherwise.

\subsection{get\_option\_type}

\begin{lstlisting}[language=Python]
def get_option_type(string key)
return PyObject_Type
\end{lstlisting}

This function takes the key in the form of a Python string. 
It raises SpudKeyError if the supplied key does not exist in the options tree.
It returns the type of the key as one of the Python object types in 3.6.2.

\subsection{get\_option\_rank}

\begin{lstlisting}[language=Python]
def get_option_rank(string key)
return int
\end{lstlisting}

This function takes the key in the form of a Python string. 
It raises SpudKeyError if the supplied key does not exist in the options tree.
It returns the rank of the key as Python ints.
0 for scalar, 1 for vector, 2 for tensor or matrix.

\subsection{get\_option\_shape}

\begin{lstlisting}[language=Python]
def get_option_shape(string key)
return (int, int)
\end{lstlisting}

This function takes the key in the form of a Python string. 
It raises SpudKeyError if the supplied key does not exist in the options tree.
It returns the shape of the key as Python two-tuple.
If the option in question is rank 1 then the second component will be -1, if the option is rank 0 (a scalar) then both entries will be -1.

\subsection{get\_option}

\begin{lstlisting}[language=Python]
def get_option(string key)
return optionvalue
\end{lstlisting}

This function takes the key in the form of a Python string. 
If key matches an option but the shape, rank or type fails to match, appropriate error code will
be set (see 3.6.1).
If key fails to match but default is present, option is set to the value of default.
If key fails to match and default is not present, the error code will be set to SpudKeyError.
It returns the value of the option, the value could be of any type.

\subsection{add\_option}

\begin{lstlisting}[language=Python]
def add_option(string key)
return None
\end{lstlisting}

This function takes the key in the form of a Python string. 
It creates a new option at the supplied key.
If the option does not currently exist, creates the option
(with data type SPUD\_NONE) and returns error code SpudNewKeyWarning.

\subsection{set\_option}

\begin{lstlisting}[language=Python]
def set_option(string key, valuetype value)
return None
\end{lstlisting}

This function takes the key in the form of a Python string, and the value to be set. 
Value could be of any type listed in 3.6.2.
If key matches an option but the shape, rank or type fails to match currently existing option,
returns an appropriate error code (see 3.6.1).
If key fails to match, creates a new option at the supplied key, sets the option to value and returns
error code SpudNewKeyWarning.
This function is for setting options in the options tree.

\subsection{set\_option\_attribute}

\begin{lstlisting}[language=Python]
def set_option_attribute(string key, valuetype value)
return None
\end{lstlisting}

This function takes the key in the form of a Python string, and the value to be set. 
Value could be of any type listed in 3.6.2.
If key matches an option but the shape, rank or type fails to match currently existing option,
returns an appropriate error code (see 3.6.1).
If key fails to match, creates a new option at the supplied key, sets the option to value and returns
error code SpudNewKeyWarning.
If key matches an option, but the existing option has children, sets the option to val and
returns error code SpudAttrSetFailedWarning.
This function is for setting options in the options tree, but additionally attempts to mark the option at the key as an attribute.

\subsection{delete\_option}

\begin{lstlisting}[language=Python]
def delete_option(string key)
return None
\end{lstlisting}

This function takes the key in the form of a Python string. 
It raises SpudKeyError if the supplied key does not exist in the options tree.
This function deletes the option in the options tree.

\subsection{print\_options}

\begin{lstlisting}[language=Python]
def print_options()
return None
\end{lstlisting}

This function takes no arguments.
It prints the entire options tree to standard output.

\chapter{Diamond}

\section{Installation}
Diamond depends on Python (version 2.3 or greater), PyGTK,
the 4Suite XML library, and the lxml library. All of these dependencies
are available in the Debian/Ubuntu repositories with the following command:
\begin{verbatim}
apt-get install python-gtk2 python-lxml python-4suite-xml
\end{verbatim}

Windows versions of these packages are available for download on their respective websites.

\section{Configuration files}

In accordance with normal Unix practice, system wide configuration for
diamond is stored in the \verb+/etc/diamond+ directory while per-user
configuration is stored in a \verb+.diamond+ directory in the user's home
directory.

\subsection{Schemas}

Diamond needs to know which Spud schemas are installed and available on the
current system. This is specified in the \verb+schemata+ subdirectory of the
\verb+/etc/diamond+ and \verb+~/.diamond+ directories. The file names in the
\verb+schemata+ directory give the filename extension associated with a
particular problem description language. The content of the file is two
lines. The first line is the name of the problem description language while
the second line contains the path of the XML syntax (\verb+.rng+) version of
the corresponding schema.

For example, the Fluidity package has a problem description language called
the Fluidity Markup Language which uses the file extension
\verb+.flml+. When installed on a system by the sysadmin, Fluidity might
create the file \verb+/etc/diamond/flml+ with the following contents:
\begin{verbatim}
Fluidity Markup Language
/usr/share/fluidity/fluidity_options.rng
\end{verbatim}
An individual user \verb+jrluser+ might have the current version of the
Fluidity svn tree checked out in their home directory and would need to
point diamond at the (posssibly updated) schema in their source
tree. \verb+jrluser+ would then create the file
\verb+/home/jrluser/.diamond/schemata/flml+ which would contain:
\begin{verbatim}
Fluidity Markup Language
/home/jrluser/fluidity/tools/fluidity_options.rng
\end{verbatim}
Now Diamond will pick up the version of the schema in \verb+jrluser+'s svn
tree rather than the version the sysadmin installed.

It is also possible to specify a URL over HTTP for the location of the
schema. For example:
\begin{verbatim}
Fluidity Markup Language
http://amcg.ese.ic.ac.uk/svn/fluidity/trunk/tools/fluidity_options.rng
\end{verbatim}
This is to facilitate centralised deployments.

\subsection{Running Diamond}

Diamond can be started in several ways.

When executed as
\begin{verbatim}
diamond
\end{verbatim}
Diamond will offer the user the choice of which registered schema to use.
If only one schema is registered, then it will use that by default.
Diamond will open a blank document of the language specified by the schema to be edited.

When executed as
\begin{verbatim}
diamond -s /path/to/schema.rng
\end{verbatim}
Diamond will open a blank document using the schema specified on the command line.

When executed as
\begin{verbatim}
diamond filename.suffix
\end{verbatim}
Diamond will inspect the registered schemas for the suffix specified
and use that schema.

\subsection{Dynamic validation}
Diamond uses the schema to guide the editing of valid documents that the schema
permits. When a blank document is opened, information necessary to make a valid
document is missing and the document is thus invalid\footnote{This is true for all
but a trivial schema which permits only one document.}. The invalidity of the document
is reflected in the colouration of the root node: valid nodes are coloured in black,
while invalid nodes are coloured in blue. As invalid nodes have their attributes
and data specified, their validity is dynamically updated and they are coloured appropriately.
The document as a whole is valid if and only if the root node is coloured black.

\chapter{Miscellaneous tools}

\section{Spud-set}

Spud-set is a short Python script which enables users to make small
modifications to the values of options in Spud options files on the command
line. This is particularly useful where users wish to script a number of
model runs exploring different values of one or more parameters.

The syntax of spud-set is:
\begin{verbatim}
spud-set filename xpath new_value
\end{verbatim}
where:
\begin{description}
\item[filename] is the name of the options file which is to be modified.
\item[xpath] is the xpath of the option to be modified (see below).
\item[new\_value] is the new value which the option should take.
\end{description}

\subsection{xpath}

An xpath is a language which provides a mechanism for addressing parts of an
xml file. It uses paths similar in some ways to those used to retrieve
options in libspud. A full description of xpath syntax is to be found in the
\href{http://www.w3.org/TR/xpath}{World Wide Web Consortium XPath
  Recommendation}. However, users of Spud will usually find it much easier
to use the xpaths generated by Diamond. The xpath for any option is shown at
the bottom of the Diamond window when that option is selected and can be
copied to the clipboard using the ``copy spud path'' option from the edit menu.

\end{document}
